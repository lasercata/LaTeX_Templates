\documentclass[a4paper, 12pt, twoside]{article}


%------------------------------------------------------------------------
%
% Author                :   Lasercata
% Last modification     :   2022.04.08
%
%------------------------------------------------------------------------


\documentclass[a4paper, 12pt, twoside]{article}


%------------------------------------------------------------------------
%
% Author                :   Lasercata
% Last modification     :   2023.11.28
%
%------------------------------------------------------------------------

%------Lang
%\usepackage[french]{babel}
%\usepackage[english]{babel}


%See https://github.com/lasercata/LaTeX_Templates for the file latex_base.sty
\documentclass[a4paper, 12pt, twoside]{article}


%------------------------------------------------------------------------
%
% Author                :   Lasercata
% Last modification     :   2023.11.28
%
%------------------------------------------------------------------------

%------Lang
%\usepackage[french]{babel}
%\usepackage[english]{babel}


%See https://github.com/lasercata/LaTeX_Templates for the file latex_base.sty
\documentclass[a4paper, 12pt, twoside]{article}


%------------------------------------------------------------------------
%
% Author                :   Lasercata
% Last modification     :   2023.11.28
%
%------------------------------------------------------------------------

%------Lang
%\usepackage[french]{babel}
%\usepackage[english]{babel}


%See https://github.com/lasercata/LaTeX_Templates for the file latex_base.sty
\input{~/Templates/latex_base.sty}


%------Circuitikz
%\usetikzlibrary{babel}             %Uncomment this to use circuitikz
%\usetikzlibrary{shapes.geometric}  % To draw triangles in trees
%\usepackage[european]{circuitikz}            %Electrical circuits drawing

%------Sections
%---To change section numbering :
% \renewcommand\thesection{\Roman{section}}
% \renewcommand\thesubsection{\arabic{subsection})}
% \renewcommand\thesubsubsection{\textit \alph{subsubsection})}

%---To start numbering sections from 0
% \setcounter{section}{-1}

%---To hide subsubsection from the table of contents (show with max depth of 2)
% \setcounter{tocdepth}{2}


%------Logo
\setlogo %Comment to remove the logo


%------Title (with default LaTeX style)
%\title{}
%\author{}
%\date{\today}

%---------------------------------Begin Document
\begin{document}
    
    \thetitle{}{}
    %\maketitle
    
    %\tableofcontents
    %\listoffigures
    %\listoftables
    %\newpage
    
    \begin{indt}{\section{}}
        .a
    \end{indt}
    
\end{document}
%--------------------------------------------End




%------Circuitikz
%\usetikzlibrary{babel}             %Uncomment this to use circuitikz
%\usetikzlibrary{shapes.geometric}  % To draw triangles in trees
%\usepackage[european]{circuitikz}            %Electrical circuits drawing

%------Sections
%---To change section numbering :
% \renewcommand\thesection{\Roman{section}}
% \renewcommand\thesubsection{\arabic{subsection})}
% \renewcommand\thesubsubsection{\textit \alph{subsubsection})}

%---To start numbering sections from 0
% \setcounter{section}{-1}

%---To hide subsubsection from the table of contents (show with max depth of 2)
% \setcounter{tocdepth}{2}


%------Logo
\setlogo %Comment to remove the logo


%------Title (with default LaTeX style)
%\title{}
%\author{}
%\date{\today}

%---------------------------------Begin Document
\begin{document}
    
    \thetitle{}{}
    %\maketitle
    
    %\tableofcontents
    %\listoffigures
    %\listoftables
    %\newpage
    
    \begin{indt}{\section{}}
        .a
    \end{indt}
    
\end{document}
%--------------------------------------------End




%------Circuitikz
%\usetikzlibrary{babel}             %Uncomment this to use circuitikz
%\usetikzlibrary{shapes.geometric}  % To draw triangles in trees
%\usepackage[european]{circuitikz}            %Electrical circuits drawing

%------Sections
%---To change section numbering :
% \renewcommand\thesection{\Roman{section}}
% \renewcommand\thesubsection{\arabic{subsection})}
% \renewcommand\thesubsubsection{\textit \alph{subsubsection})}

%---To start numbering sections from 0
% \setcounter{section}{-1}

%---To hide subsubsection from the table of contents (show with max depth of 2)
% \setcounter{tocdepth}{2}


%------Logo
\setlogo %Comment to remove the logo


%------Title (with default LaTeX style)
%\title{}
%\author{}
%\date{\today}

%---------------------------------Begin Document
\begin{document}
    
    \thetitle{}{}
    %\maketitle
    
    %\tableofcontents
    %\listoffigures
    %\listoftables
    %\newpage
    
    \begin{indt}{\section{}}
        .a
    \end{indt}
    
\end{document}
%--------------------------------------------End



\usetikzlibrary{babel}
\usetikzlibrary{shapes.geometric}
\usepackage{circuitikz}


%---------------------------------Begin Document
\begin{document}
    
    %For dark mode :
    % \pagecolor{black}
    % \color{white}
    
    \thetitle{Ti$k$z}{electrical circuits}
    
    %\tableofcontents
    %\newpage

    \begin{circuitikz}
        \draw
        (0,0) to[R=R, o-o] (2,0)
        (4,0) to[vR=vR, o-o] (6,0)
        (0,2) to[transmission line=transmission line, o-o] (2,2)
        (4,2) to[closing switch=closing switch, o-o] (6,2)
        (0,4) to[european current source=european current source, o-o] (2,4)
        (4,4) to[european voltage source=european voltage source, o-o] (6,4)
        (0,6) to[empty diode=empty diode, o-o] (2,6)
        (4,6) to[full led=full led, o-o] (6,6)
        (0,8) to[generic=generic, o-o] (2,8)
        (4,8) to[sinusoidal voltage source=sinusoidal voltage source, o-o] (6,8)
        ;
    \end{circuitikz}

    \begin{circuitikz}
        \draw (0,0)
        to[V,v=$U_q$] (0,2) % The voltage source
        to[nos] (2,2)
        to[generic=$R_1$] (2,0) % The resistor
        to[short] (0,0);
        \draw (2,2)
        to[short] (4,2)
        to[L=$L_1$] (4,0)
        to[short] (2,0);
        \draw (4,2)
        to[short] (6,2)
        to[C=$C_1$] (6,0)
        to[short] (4,0);
   \end{circuitikz}
   
   code :
   
   \begin{lstlisting}[language=tex]
\begin{circuitikz}
    \draw (0,0)
    to[V,v=$U_q$] (0,2) % The voltage source
    to[nos] (2,2)
    to[generic=$R_1$] (2,0) % The resistor
    to[short] (0,0);
    \draw (2,2)
    to[short] (4,2)
    to[L=$L_1$] (4,0)
    to[short] (2,0);
    \draw (4,2)
    to[short] (6,2)
    to[C=$C_1$] (6,0)
    to[short] (4,0);
\end{circuitikz}
   \end{lstlisting}
   
    \begin{circuitikz}
        \draw (0,0) to [capacitor=$C$, o-] (2, 0)
        to [L=$L$] (2, -2)
        to [short, -o] (0, -2);
        \draw (2, -2) to [short] (4, -2)
        to [generic=$R$] (4, 0)
        to [short] (2, 0);
        \draw[->] (0, -1.5) -- node[right] {$u_e$} (0, -.5);
        \draw[->] (4.5, -1.5) -- node[right] {$u_s$} (4.5, -.5);
    \end{circuitikz}
    
    code :
    
    \begin{lstlisting}[language=tex]
\begin{circuitikz}
    \draw (0,0) to [capacitor=$C$, o-] (2, 0)
    to [L=$L$] (2, -2)
    to [short, -o] (0, -2);
    \draw (2, -2) to [short] (4, -2)
    to [generic=$R$] (4, 0)
    to [short] (2, 0);
    \draw[->] (0, -1.5) -- node[right] {$u_e$} (0, -.5);
    \draw[->] (4.5, -1.5) -- node[right] {$u_s$} (4.5, -.5);
\end{circuitikz}
    \end{lstlisting}
    
        \begin{figure}[h!]
        \begin{center}
            \begin{circuitikz}
                \draw (0,0)
                to[V,v=$U_q$] (0,2) % The voltage source
                to[short] (2,2)
                to[R=$R_1$] (2,0) % The resistor
                to[short] (0,0);
            \end{circuitikz}
            \caption{My first circuit.}
        \end{center}
    \end{figure}


    \begin{circuitikz}
        \draw (0, 0) to [lamp] (0, 4)
        ;
    \end{circuitikz}
    
    \begin{circuitikz}
        \draw
        (0,0) to[battery] (0,4)
        to[ammeter] (4,4) 
        to[C] (4,0) -- (3.5,0)
        to[lamp, *-*] (0.5,0) -- (0,0)
        (0.5,0) -- (0.5,-2)
        to[voltmeter] (3.5,-2) -- (3.5,0)
        ;
    \end{circuitikz}
    
    
    
\end{document}
%--------------------------------------------End
